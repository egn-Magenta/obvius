\documentclass[a4paper,12pt]{article}

\usepackage[latin1]{inputenc}
\usepackage[T1]{fontenc}
\usepackage[dvips]{graphicx}

\usepackage{array}

\usepackage{fancyheadings}
\pagestyle{fancy}
\lhead{}
\rhead{}
\lfoot{{\tiny \today}}
\rfoot{{\tiny asjo@magenta-aps.dk}}
\setlength{\headrulewidth}{0pt}

%\linespread{1.1}
\setlength{\parindent}{0pt}
\setlength{\parskip}{1ex plus 0.5ex minus 0.2ex}

\newcommand{\obvius}{\textsc{Obvius}}
\newcommand{\modperl}{\texttt{mod\_perl}}
\newcommand{\modrewrite}{\texttt{mod\_rewrite}}
\newcommand{\modproxy}{\texttt{mod\_proxy}}
\newcommand{\notfound}{\texttt{NOT\_FOUND}}
\newcommand{\forbidden}{\texttt{FORBIDDEN}}

\newenvironment{quirk}{\begin{quote}\emph{Quirk:}}{\end{quote}}

\title{\obvius\\{\Large The journey of a request}}
\date{}
\author{}

\begin{document}

\maketitle 

\begin{abstract}
  This document describes the typical flow of a request through the
  \obvius\ Content Management system.
\end{abstract}

\pagebreak
\tableofcontents
\pagebreak


\section{Introduction}

The \obvius\ content management system is based primarily on the
webserver Apache\footnote{http://httpd.apache.org/} with the {\modperl}
module\footnote{http://perl.apache.org/} for perl-integration,
Mason\footnote{http://masonhq.com/} as a template-system -- and for
implementing the administration system -- and
MySQL\footnote{http://mysql.com/ -- Please note
  that due to the use of the perl DBI abstraction layer and
  DBIx::RecordSet, using another database consists primarily of
  converting the schema.} for storage.

This document describes the transfer of control through \obvius\ during
a request, hopefully making it easier to pinpoint where to look when
developing with \obvius, and when debugging.


\section{Apache}

\subsection{A browser makes a request}

When a browser makes a request, Apache locates a non-occupied
child-process and passes the request on to that.

The child-Apache looks at the supplied \texttt{Host:}-header and
decides what virtual host to serve the request from.

Each website is included in the Apache-configuration by its
\texttt{conf/site.conf}-file.

\subsubsection{site.conf}

An Apache-configuration file that defines the site. Includes
\texttt{names.conf}, ``runs'' the perl-file \texttt{setup.pl}
(once, on start) and includes the Apache-configurationfile
\texttt{setup.conf}.

\subsubsection{names.conf}

Defines the \texttt{ServerName} and any \texttt{ServerAliases}. In a
file by itself, so if the server is set up in a two-layer
Apache-configuration the \texttt{front.conf} can share the names
easily (see also appendix \ref{proxying}).

\subsubsection{setup.conf / cache}

\texttt{setup.conf} contains mainly the required
\modrewrite-directives to serve static files, serve files from the
cache and invoke the correct perl-methods according to what part of
the website the request is destined for (admin or public).

It also contains the boring details of log-files, expires-information
for static-files and so on.

\subsubsection{setup.pl}

This perl startup-file reads the configuration-file located in
\texttt{/etc/obvius/example.conf} and creates the three
website-specific objects that each \obvius\ website uses:

\begin{description}
\item[\texttt{Example::Site::Public}] --- used for handling the public website.
\item[\texttt{Example::Site::Admin}]  --- used for handling the administration interface.
\item[\texttt{Example::Site::Common}] --- used by both for generating the common
  content.
\end{description}

\subsubsection{front.conf}

Is used to run a bunch of small \modproxy-enabled Apache-processes
that serve static files and cached files, while proxying other
requests to another bunch of \modperl\ enabled Apache-processes.


\section{Perl}

If the request isn't for a static file or can be served from the
cache, the request is handled by either the \texttt{/public-}, or the
\texttt{/admin} \texttt{Location}-block in \texttt{setup.conf}.

Using the {\modperl} directives \texttt{PerlAccessHandler},
\texttt{PerlAuthenHandler}, \texttt{PerlAuthzHandler} and
\texttt{PerlHandler}, control is passed to the appropriate methods on
one of the websites two site-objects, \texttt{Example::Site::Public}
or \texttt{Example::Site::Admin}.


\subsection{Handlers}

Usually\footnote{That is, on all websites.} the websites objects
inherit from the global \texttt{WebObvius::Site::Mason} class.

\subsubsection{authen\_handler (admin)}

Checks whether the user is logged in, and if not uses http basic
authentification to allow the user to do so.

Whether the username/password is correct is checked by calling the
\texttt{obvius\_connect}-method on the site object.

% \subsubsection{authz\_handler (admin)}
% Isn't used, just calls the access_handler... don't ask me why (I
% don't know).

\subsubsection{access\_handler (admin and public)}

This is where missing trailing slashes are added (by way of
redirection).

\begin{quirk}
  If the URI ends in \texttt{.html/}, the trailing slash is \emph{removed!}
\end{quirk}

% Too detailed?
%   The \obvius\ object is obtained by calling the
%   \texttt{obvius\_connect} method on the siteobject.

The document is looked up
% Too much detail?
%   using the \texttt{obvius\_document} method on the site-object
and a {\notfound} message is returned to Apache if the
lookup fails.

% Too detailed?
%   If the document is found it is stored on \texttt{pnotes} along with
%   the site object, for further access.

\subsubsection{handler (admin and public)}

This is the main attraction, where a page or some data are generated
and returned to the browser.

\textbf{public}

If the document isn't public, {\notfound} is returned.

If the document is expired, {\forbidden} is returned.

If the document-type-object's \texttt{alternate\_location}-method
returns something, the browser is immediately redirected to that
something.

If the document-type-object's \texttt{raw\_document\_data}-method
returns something, then \obvius\ sets the appropriate headers, cache
the raw data, if possible, and returns the raw data directly to the
browser.

(This part isn't executed for admin, because in admin we need to be
able to edit documents that redirect or return raw data -- so these
special cases are handled in the admin-Mason application instead).

\textbf{both}

Mason is executed.

If everything went okay, various headers are added, the page is added
to the cache, if possible (it never is in admin), and the HTML is then
returned to the browser.


\section{Mason}

\emph{input- and output-objects, overall doctype/template-concept}

\subsection{switch}

\subsubsection{subsite}


\section{Cache}


\section{Headers}


\pagebreak
\appendix
\section{Proxying}\label{proxying}

In anything but the most basic, low-traffic setup, it is highly
recommended to follow the recommendations of the \modperl-developers,
and set up a large number of tiny Apache-processes to handle the
simple tasks and proxy more complicated tasks through to a much lower
number of large \modperl-equipped
Apache-processes\footnote{http://perl.apache.org/docs/1.0/guide/strategy.html
  -- especially the section ``Adding a Proxy Server in http Accelerator Mode''.}

Fortunately this is relatively easy with \obvius. In the
\texttt{conf/}-dir of each website is a configuration-file called
\texttt{front.conf} that the tiny Apache-setup uses for configuring
the proxying.

The tiny Apache's listen on port 80 and the large Apache's are then
configured to listen on port 81.

The tiny Apache's will handle static files found in the
\texttt{docs/}-dir and pages present in the cache -- thus typically
serving more than two thirds of the total number of hits on a website.
If the request can't be handled by the tiny Apache (the page is not in
the cache, or can't be cached), {\modproxy} passes the request to one
of the large processes, which then handles the request.

By default logging is doubled -- the tiny as well as the large
Apache's log requests. The large Apache's logfiles are not that
interesting, because they register the IP-address of the tiny
Apache's; always the same. Statistics should be generated from the
tiny Apache's logs (\texttt{front\_custom.log}).


\end{document}